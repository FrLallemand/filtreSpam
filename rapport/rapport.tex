\documentclass[a4paper, 11pt]{article}
\usepackage[utf8]{inputenc}
\usepackage[T1]{fontenc}
\usepackage{lmodern}
\usepackage{graphicx}
\usepackage[french]{babel}
\usepackage{listings}
\usepackage{tikz}
\usepackage{enumitem}

\title{Projet : filtre anti-spam}
\author{Ladeveze Quentin\\Lallemand François}
\begin{document}
\maketitle


\section{Utilisation du programme}
Les sources de ce projet se trouvent dans le fichier filtreAntiSpam.java, à compiler par la commande:
\begin{lstlisting}[language=bash]
  javac filtreAntiSpam.java
\end{lstlisting}
La classe filtreAntiSpam se lance par les commande:
\begin{itemize}[label=\textbullet]
\item java filtreAntiSpam <test\_folder> <number of spam in test folder> <number of ham in test folder>
  Effectue un test avec les fichiers situés dans <test\_folder>, sur le nombre de SPAM et de HAM indiqués. Le programme demande ensuite un nombre de SPAM et de HAM dans la base d'apprentissage et lance le test.
\item java filtreAntiSpam save <file where to store classifier > <number of spam> <number of ham>
  Cette option effectue un apprentissage sur le nombre de SPAM et de HAM indiqués et enregistre le classifieur dans le fichier dont le nom est donné en paramètre.
\item java filtreAntiSpam filtre\_ligne <source file for the classifier > <message> <HAM or SPAM>
  Cette option effectue un apprentissage en ligne en se basant sur un classifieur existant.
\item java filtreAntiSpam <source file for the classifier > <message to test>
  Cette option permet de tester un mail rapidement à partir d'un classifieur existant.
\end{itemize}

\section{Répartition du travail}
Nous nous sommes répartis le travail de la manière suivante :
Nous avons réfléchi ensemble aux questions avant d'implémenter les réponses sur une seule machine.

\section{Choix effectués}
Afin de décider si un message est ou non un spam, plutôt que de calculer :
\[ \prod\limits_{j=1}^d {(b^jSPAM)^x}^j {(1-b^jSPAM)^{1-x}}^j \]
Nous avons utilisé les logarithmes afin de calculer :
\[ \sum\limits_{j=1}^d ln({(b^jSPAM)^x}^j) ln({(1-b^jSPAM)^{1-x}}^j) \]

Les \[ x^j \] et \[ 1- x^j \] sont représentés sous forme de booléens, afin de simplifier les calculs du classifieurs en utilisant des conditions ``if'' plutot que des calculs de puissance.

La fonction lire\_message sépare le texte selon l'expression régulière ``\\ W+''. W correspond aux charactères qui ne sont ni des chiffres, ni des lettres. Cela permet de reconnaitre les mots suivis d'un signe de ponctuation, comme ``House,''.

\section{Exemple d'exécution}

\begin{lstlisting}[language=bash]
java filtreAntiSpam  basetest 100 200
Nombre de spams dans la base d'apprentissage : 200

Nombre de hams dans la base d'apprentissage : 200

Apprentissage sur: 200 spams et 200 hams...
=== TEST ===

P(Y=SPAM | X=x) = -291,529731, P(Y=HAM | X=x) = -319,106067.
Spam number 0 classified as SPAM
P(Y=SPAM | X=x) = -295,396522, P(Y=HAM | X=x) = -338,647222.
Spam number 1 classified as SPAM
P(Y=SPAM | X=x) = -296,500981, P(Y=HAM | X=x) = -350,242648.
Spam number 2 classified as SPAM
P(Y=SPAM | X=x) = -296,500981, P(Y=HAM | X=x) = -350,242648.
Spam number 3 classified as SPAM
P(Y=SPAM | X=x) = -205,249638, P(Y=HAM | X=x) = -253,033832.
...
P(Y=SPAM | X=x) = -99,320038, P(Y=HAM | X=x) = -87,989156.
Spam number 29 classified as HAM *** error ***
...
P(Y=SPAM | X=x) = -103,779099, P(Y=HAM | X=x) = -100,698392.
Spam number 97 classified as HAM *** error ***
P(Y=SPAM | X=x) = -186,837117, P(Y=HAM | X=x) = -196,483656.
Spam number 98 classified as SPAM
P(Y=SPAM | X=x) = -220,585287, P(Y=HAM | X=x) = -248,344009.
Spam number 99 classified as SPAM
...
P(Y=SPAM | X=x) = -386,925411, P(Y=HAM | X=x) = -369,281157.
Ham number 0 classified as HAM
P(Y=SPAM | X=x) = -168,757847, P(Y=HAM | X=x) = -143,911920.
Ham number 1 classified as HAM
...
P(Y=SPAM | X=x) = -417,327559, P(Y=HAM | X=x) = -431,827431.
HAM number 54 classified as SPAM *** error ***
P(Y=SPAM | X=x) = -466,173949, P(Y=HAM | X=x) = -474,356725.
HAM number 55 classified as SPAM *** error ***
...
P(Y=SPAM | X=x) = -174,426532, P(Y=HAM | X=x) = -155,199404.
Ham number 196 classified as HAM
P(Y=SPAM | X=x) = -255,184287, P(Y=HAM | X=x) = -236,380796.
Ham number 197 classified as HAM
P(Y=SPAM | X=x) = -130,871171, P(Y=HAM | X=x) = -118,576840.
Ham number 198 classified as HAM
P(Y=SPAM | X=x) = -368,254599, P(Y=HAM | X=x) = -340,982368.
Ham number 199 classified as HAM

Test error on 100 spams : 12.0%
Test error on 200 hams : 2.0%
Global test error : 5.333333333333333%


\end{lstlisting}

\end{document}
